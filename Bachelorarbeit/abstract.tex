\chapter*{Abstract}
\thispagestyle{empty}
%
During the development of high-voltage direct current transmission systems, all test documents are included in a so-called Master Test Plan. These tests begin with factory tests of individual components and extend to acceptance tests at the customer's plant. This plan is currently based on an Excel-Sheet. It is processed by several departments over the course of the project. Since this process makes the plan extremely unclear and the multiple edits make it unstructured, it is to be transferred to an Application Lifecycle Management Tool. An associated workflow is also to be specified. In addition, an import and export mask is to be introduced. On top of this, the IEC 81346 and IEC 61355 standards have to be observed. After getting used to the online tool and the Master Test Plan, the first task to perform, was to define a structure for the converted plan. To now proceed, the variables and the input masks had to be created. Additionally some JavaScript scripts for simplyfing the use of the UI were implemented, based on a existing Polarion Extension. During the implementation a plugin has been developed aswell. It executes a multiplication algorhytm based on the engineer standards mentioned above. 
The tests show, that Polarion ALM is a good way to manage plant data, but not for handling processing and multiplying many items. In addition some problems with the official manuals occured and the project is not working as it should. It would be advisable to process the data in an external tool before importing it to Polarion.
%

\newpage

\chapter*{Zusammenfassung}
\thispagestyle{empty}
%
Bei der Entwicklung von Hochspannungs-Gleichstrom-Übertragungssystemen werden alle Prüfdokumente in einen so genannten Master-Prüfplan aufgenommen. Diese Tests beginnen mit Werkstests einzelner Komponenten und reichen bis hin zu Abnahmetests auf der Anlage des Kunden. Dieser Plan basiert derzeit auf einer Excel-Tabelle. Er wird im Laufe des Produktzykluses von mehreren Abteilungen bearbeitet. Da dieser Prozess den Plan extrem unübersichtlich und durch die mehrfachen Bearbeitungen unstrukturiert macht, soll er in ein Application Lifecycle Management Tool überführt werden. Ein zugehöriger Workflow soll ebenfalls implementiert werden. Darüber hinaus soll eine Import- und Exportmaske eingeführt werden. Zusätzlich sind die Normen IEC 81346 und IEC 61355 zu beachten. Nach der Einarbeitung in das Online-Tool und den Master-Testplan war die erste Aufgabe, eine Struktur für den konvertierten Plan zu entwickeln. Um nun fortzufahren, mussten die Variablen und die Eingabemasken erstellt werden. Zusätzlich wurden einige JavaScript-Skripte zur vereinfachten Nutzung der Benutzeroberfläche entwickelt, die auf einer bestehenden Polarion-Erweiterung basieren. Im Zuge der Implementierung wurde auch ein Plugin entwickelt. Es führt einen Multiplikationsalgorithmus aus, der auf den oben erwähnten Normen basiert. 
Nach dem Testen war klar, dass Polarion ALM eine gute Möglichkeit ist, Anlagendaten zu verwalten, aber nicht für die Verarbeitung und Multiplikation vieler Elemente geeignet ist. Darüber hinaus traten einige Probleme mit den offiziellen Handbüchern auf und das Projekt funktioniert nicht so, wie es sollte. Es ist ratsam, die Daten in einem externen Programm zu verarbeiten, bevor sie nach Polarion importiert werden.
%