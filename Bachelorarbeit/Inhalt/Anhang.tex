\appendix
\pagenumbering{roman}
\newgeometry{left=35mm, right=35mm, top=20mm, bottom=20mm}
%%%
\chapter{Anhang}
%
\thispagestyle{empty}
\pagestyle{empty}
%%
\begin{landscape}
\section{Dokumentationsschema nach \cite{12}(S.47-50)}
\label{sec:schema}
%%
%%
\pic{schema1}{1.5}{Schema zur Dokumentation für Leittechnik Objekte off-site}{schema1}
\pic{schema2}{1.5}{Schema zur Dokumentation für Leittechnik Objekte on-site}{schema2}
\pic{schema3}{1.5}{Schema zur Dokumentation für nicht-Leittechnik Objekte off-site}{schema3}
\pic{schema4}{1.5}{Schema zur Dokumentation für nicht-Leittechnik Objekte on-site}{schema4}
%%
\section{Polarion-Oberfläche}
\label{sec:polarionui}
%%
\pic{polarion}{1.5}{Gesamte Benutzeroberfläche}{polarionui}
%%
\end{landscape}
%%
\section{Velocity und API Code-Beispiel}
\label{sec:veloapi}
%%
\begin{figure}[H]
\begin{lstlisting}[language=Java]
## 
## start transaction 
## 

$transactionService.beginTx() 
  
## 
## create a new workitem 
## 
#set($newWi=$trackerService.createWorkItem($projectService.getProject($page.project))) 
  
## set the workitem type 
#set($enumTypes=$trackerService.getTrackerProject($page.project).workItemTypeEnum) 
$newWi.setType($enumTypes.wrapOption("task")) 
  
## set the title 
$newWi.setTitle("Created by Velocity") 
  
## set the description 
#set($textObject=$objectFactory.newPlainText("This workitem was created by Velocity")) 
$newWi.setDescription($textObject) 
  
## save the workitem 
$newWi.save() 
  
## 
## end transaction, persist the work item changes 
## 
$transactionService.commitTx()
\end{lstlisting}
\caption{Codebeispiel Velocity}
\label{fig:bspvelo}
\end{figure} 
%%
Mit \inline{\#set()} werden Variablen gesetzt. Um eine Änderung an Workitems persistent zu machen, muss immer eine \inline{transaction} begonnen und nach dem Speichern des Workitems beendet werden.\\
Die Extension um die Instanz \inline{$objectFactory} zu verwenden, ist unter der verwendeten Polarionversion nicht mehr verfügbar.
%%
\section{Interface Testdocument}
\label{sec:testdocuinter}
\FloatBarrier
%%
 
\pic{workitem-405-1}{1}{Testdocument Teil 1}{testdocument1}
\newpage
%%
\pic{workitem-405-2}{1}{Testdocument Teil 2}{testdocument2}
%%
\begin{landscape}
\section{Importpreview}
\label{sec:import}
%%
\pic{preview}{1.8}{Preview eines Imports}{preview}
%%
\end{landscape}
\section{Fehlermeldung bei fehlerhaften Workitems}
\label{sec:error}
%%
\pic{error}{1}{Fehlermeldung/Exceptionstack}{error}
%%

%\unappendix
%end of appendix
\restoregeometry