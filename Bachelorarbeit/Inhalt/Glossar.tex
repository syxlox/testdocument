\chapter*{Glossar}
%
\addcontentsline{toc}{chapter}{Glossar}
%
\addtocontents{toc}{\cftpagenumberson{chapter}}

\label{cha:glossar}
\thispagestyle{empty}
\vspace{1cm}

\begin{acronym}[HTML] %%inside the square bracket, write the acronym, which is visually the biggest
	\acro{ac}[AC]{Alternating Current}
	\acro{alm}[ALM]{Application Lifecycle Management}
	\acro{api}[API]{application programming interface}
	\acro{dc}[DC]{Direct Current}
	\acro{dcc}[DCC]{Document Classification Code}
	\acro{dms}[DMS]{Dokumenten Managment System}
	\acro{hgü}[HGÜ]{Hochspannungsgleichstromübertragung}
	\acro{html}[HTML]{Hypertext Markup Language}
	\acro{id}[ID]{Identifikator}
	\acro{ide}[IDE]{integretated development environment}
	\acro{lcc}[LCC]{line-commutated current-sourced converters}
	\acro{mtp}[MTP]{Master Test Plan}
	\acro{mtm}[MTM]{Master Test Matrix}
	\acro{otc}[OTC]{Object Type Catalogue}
	\acro{pam}[PAM]{Parameter Manager}
	\acro{ui}[UI]{Userinterface}
	\acro{vsc}[VSC]{Voltage-Source Converter}
	\acro{vtl}[VTL]{Velocity Template Language}
	\acro{xml}[XML]{Extensible Markup Language}
\end{acronym}